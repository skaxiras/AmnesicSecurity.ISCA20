\begin{verbatim}
    1 Problem statement: 
        Spectre, Meltdown, ++?
    2 Summary of recent arch. solutions
        Making speculative data invisible
        2.1 Overhead & complexity per se
        2.2 Limitations
        2.3 Coverage
    3 Proposed solution: 
        Recompute speculative load values
        3.1 Overhead & complexity 
            Put into perspective
        3.2 Limitations 
        3.3 Coverage
        3.4 Putting it all together:
            * Orthogonal to which, 
                alternative to what, ... wrt 2
            * Open new vulnerabilities?
\end{verbatim}


 Side-channel attacks come in many different forms exploiting all kinds of information stored in ---\emph{what should normally be software-invisible}--- microarchitectural structures. Attacks have been demonstrated using a wide variety of measurements, including electromagnetic emissions~\cite{agrawal2002side}, energy consumption~\cite{kocher1999differential}, and even input/output devices\cite{genkin2014rsa,ferrigno2008aes,carmon2017photonic}. The most common, practical (and first to be discovered~\cite{bernstein2005cache} \textcolor{blue}{Unclear what first refers to as~\cite{agrawal2002side} is older. First of those that are practical?}) type of attack is based on \emph{timing}. More specifically, timing attacks infer secrets by measuring time differences resulting from microarchitectural optimizations.

The classic example of a timing side channel are processor caches, which store the address and value of recently accessed memory locations. Accesses to cached memory locations are faster, revealing information about which addresses were or were not recently accessed~\cite{yarom_flush+_2014,liu15llc,irazoqui_cross_2016}. Other types of side channels are also possible involving a variety of microarchitectural state (e.g., coherence states, directories, DRAM row buffers) 
and various attacks have been demonstrated in practice~\cite{}.

With the disclosure of Spectre~\cite{kocher_spectre_2018} and Meltdown~\cite{lipp_meltdown_2018} in early 2018, \emph{speculation}, one of the fundamental techniques for achieving high
performance, proved to be a significant security hole, leaving the door wide
open for side-channel attacks to ``see'' protected data~\cite{kocher_spectre_2018,lipp_meltdown_2018}.
As far as the instruction set architecture (ISA) and the target program are
concerned, leaking information across a covert timing side-channel (based on microarchitectural state and structures) is not illegal
because it does not violate the functional, architectural behavior of the program.
But \emph{speculative} side channel attacks reveal secret information during \emph{misspeculations}, 
i.e., \emph{discarded execution that is not a part of the normal execution of a program}.
The stealthy nature of a speculative side-channel attack is based on
\emph{microarchitectural} state being changed by misspeculation even when the \emph{architectural}
state is not.

\noindent \textbf{First response techniques: hide\&replay, delay, or cleanup?}
The architecture community responded to the threat of speculative side-channel attacks by proposing a number of techniques to prevent speculative execution from revealing secrets. These techniques fall into one of the following three broad categories:
%
\squishlist
\item{\textbf{Hide\&Replay:}} During speculation, perform memory accesses in a manner that does not perturb any microarchitectural state ($\mu$-state) in the memory system (except DRAM row buffers which is unavoidable \textcolor{blue}{Doesn't a closed page policy avoid this problem?}); subsequently, perform a replay of the access (when it becomes non-speculative) to affect the correct changes in the $\mu$-state~\cite{yan_invisispec:MICRO2018,sakalis2019ghost}. 

\item{\textbf{Delay:}} Delay speculative changes in $\mu$-state until execution is non-speculative. Sakalis et al. proposed to delay loads that miss in the L1 (\emph{Delay on miss}) until they are non-speculative~\cite{sakalis2019efficient}. This delays any $\mu$-state change in the memory hierarchy. A different form of delay (\emph{NDA}) proposed by Weisse et al., is to prevent speculative data propagation by delaying \emph{dependent instructions} from executing with speculative inputs~\cite{weisse2019nda}.

\item{\textbf{Cleanup:}} Perform speculative changes in $\mu$-state but then \emph{undo} if speculation is squashed~\cite{saileshwar2019cleanupspec}.
\squishend

Unfortunately, all of these were proposed under different threat models and different assumptions about the security mechanisms throughout the system, so their contribution to increased security and their cost (in performance, area, and energy) are \emph{not directly comparable}. \autoref{sec:rel} provides an extensive discussion. Our own tests show that when some of the techniques (e.g., Delay-on-miss and NDA) are examined under the same conditions (threat models) yield comparable performance results, but significantly more work is required to be able to arrive at meaningful comparisons for the whole of the area.

In this work we assume a threat model that permits all types of speculation and focuses on timing side-channels of the complete memory hierarchy (including DRAM). This corresponds to the threat model of 
Delay-on-miss~\cite{sakalis2019efficient}. Other threat models are beyond the scope of this paper {\color{red} but we make an effort to compare where possible}. %Threat model: ALL SHADOWS/ALL HIERARCHY. This is the threat model for Delay-on-miss, but no other proposed technique (NDA:  C-SHADOWS/ALL-$\mu$-state, Invisispec: ALL-SHADOWS/ALL-HIERARCHY-MINUS-DRAM, CleanupSpec ALL-SHADOWS/L1)

\noindent \textbf{It's the MLP ... :} Delay-on-miss is the simple idea of delaying any speculative load until the earliest time when it becomes non-speculative. In all its simplicity, Delay-on-miss, does a remarkable job of preserving a significant part of the performance of the unprotected baseline~\cite{sakalis2019efficient}. This is because, contrary to intuition, it does not destroy all memory-level-parallelism (MLP), as we explain in Section \ref{sec:back}.
In addition to Delay-on-miss, Sakalis et al. propose to use \emph{value prediction} (\emph{VP}) for the delayed misses in hope of performing useful work (saving latency) for the delayed loads and their dependent instructions. But here is the pitfall of this approach: VP must still be \emph{validated in-order} after the predicted loads become non-speculative. This means that the validation of these loads \emph{cannot have any MLP}.\footnote{This is irrespective of whether we can non-speculatively reorder loads for consistency reasons~\cite{aros-isca17}---See Section \ref{sec:back}.} {\color{red} Our results (based on a detailed simulation model of the proposed techniques) show that VP---even at its theoretical limits in coverage and accuracy---has very little to offer on top of Delay-on-miss.}

\noindent \textbf{Our approach:} It is exactly this problem that we address in this paper. We ask the question: Can we create ``secret'' values, \emph{invisible to an attacker}, for the delayed loads, out of thin air without having to compromise MLP to validate them afterwards? The key realization of this paper is that the answer lies in \emph{value re-computation} (\emph{RC}\footnote{Not to be confused with Release Consistency (also RC)---we will make an effort to clarify what we mean.}) also known as \emph{Amnesiac Computing}~\cite{}.

A summary description of the approach.

\noindent \textbf{Our Contributions:} {\color{red} TO BE REVISED!}
\squishlist
\item We identify the fundamental weakness (MLP) concerning a state-of-the-art/recent proposal for value prediction 
\item We propose to apply an unconventional idea, \emph{value recomputation} (previously proposed as a means to evade the cost of moving data in the memory hierarchy) to solve this problem. We develop a new technique/version/solution/method/... for value recomputation well fitted to the security problem at hand and show the synergy with Delay-on-miss.
\item We put the new solution in context with related approaches as allowed by the differences in the threat models of each proposal.
\squishend

\noindent \textbf{A summary of our results:} Blah blah blah.






