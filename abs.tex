\begin{verbatim}
    Brief problem statement: 
        Spectre, Meltdown, ++?
    Brief summary of recent arch. solutions: 
        Making speculative data invisible
    Proposed solution: 
        Recompute speculative load values
        Crisp description of high-level idea
        Pros, cons wrt related work
        Catchy evaluation numbers
\end{verbatim}

Recent architectural approaches to address speculative side-channel attacks aim to prevent software from detecting any changes %hide changes 
in the microarchitectural state while in speculation. 
%
%\emph{delay} changes until non-speculative, \emph{hide} changes and \emph{replay} when non-speculative, or \emph{cleanup} changes on mispeculation. In this work we focus on work that \emph{delays} changes until non-speculative.
One approach is to \emph{delay} micro-architectural changes until they are non-speculative. A \emph{Delay-on-miss} technique, simply delays loads that miss in the L1 until they become non-speculative, thus ensuring that transient changes in the memory hierarchy are not possible.
%
To recover some of the latency, while waiting for the loads to become non-speculative, \emph{Value Prediction} (VP) of the delayed-load's value has been proposed.

\textcolor{blue}{Too specific, sounds like fixing one particular problem, i.e., delay on miss and VP, instead of tackling the larger problem.}

We observe that introducing a new kind of speculation to hide other speculation, has the unfortunate effect of \emph{serializing} the execution (VP-validation) of all loads that are value-predicted, \emph{even when such serialization is not imposed by the implementation of a memory model}. \emph{While loads that are value-predicted may contribute a little to ILP, their serialization detracts from MLP.} Because of this, the contribution of value prediction in the end performance is fundamentally limited.

Our insight is that we can achieve the same goal as value prediction (increasing ILP by early-performing some of the loads that miss) but without incurring its negative side-effect (serialization of validation and hindering of MLP) if we can \emph{non-speculatively} deliver load values on demand. 

We achieve this with \emph{Value Recomputation}, which trades computation for data transfer. This technique was previously proposed in an entirely different context: to reduce energy-expensive data transfers in the memory hierarchy. If we can safely, non-speculatively, recompute a value in isolation (without being seen from the outside), we do not expose any information that we would do otherwise, by transferring such value via the memory hierarchy.

In this paper we demonstrate the potential of recomputation in relation to a Delay-on-miss approach of hiding speculation and show that ... Results Results Results. Further, {\color{red} in contrast to VP where one pays a cost (energy, area) in hope of recovering some of the lost performance, recomputation is not only providing better performance than VP, but at the same time delivers its natural benefit which is an absolute reduction in data transfers in the memory hierarchy, hence an overall reduction in energy. This makes our proposal the first in which a more secure architecture surpasses the baseline in terms of energy-efficiency (EDP).}




