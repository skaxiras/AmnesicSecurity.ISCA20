\subsection{Threat Model}
\label{sec:threat}
%\begin{verbatim}
  * Formal definition
  * Catchy motivating example?
 \end{verbatim} We do not need more sub-levels do we?

\subsection{Delay on Miss and Value Prediction}
\label{sec:dom-vp}
%\input{dom-vp} We do not need more sub-levels do we?
Delay-on-miss and VP. Delay-on-miss preserves some MLP. VP on the other hand destroys MLP.

\paragraph{Why is Delay on Miss so good to begin with}
Explain that we gain a lot out of intra-cache-line MLP.
If possible give a breakdown of intra-cache-line MLP vs. inter-cache-line MLP.
We lose the second and that's what we experience as performance loss.

\paragraph{Why VP cannot offer much on top}
VP needs to be validated. In order! Non-speculative load-load reordering cannot hep here as we are bound by a new speculation: VP-speculation (VP-Shadow). This prevents any MLP for the validation phase. Whatever VP wins in latency (pre-executing load and dependent instructions if correct), looses later in MLP. Result: limited.

\subsection{Data Recomputation}
\label{sec:recmp}
%\begin{verbatim}
    Summarize Amnesiac
        [u-arch details go to main sect.]
    Put into perspective 
        "Security-aware" recomputation
        Cmp & Contrast
            Slice generation policy
        Vulnerabilities?
\end{verbatim}
 We do not need more sub-levels do we?
Proposed as a way to trade in-core computation (cheap in energy) to data movement (expensive in energy) and at the same time perform better when the latency to re-compute is less than the latency to move data throughout the memory hierarchy.
\paragraph{Why RC can offer beyond VP}
Benefits in both latency and MLP: Latency: if recomputation is faster than a miss. MLP: No need to validate, hence does not impose any further restrictions on MLP, hence better performance. Energy: additional benefits in energy if recomputation needs less energy than data movement. Limitation: cannot do too much of it in practice.



 