\subsection{Threat Model}
\label{sec:threat}
%\begin{verbatim}
  * Formal definition
  * Catchy motivating example?
 \end{verbatim} We do not need more sub-levels do we?
{ \color{red} [FIXME: Probably discuss after the shadows?] }

\subsection{Speculation Shadows} 
Sakalis et al. introduced the concept of \textbf{\emph{Speculative Shadows}} to reason about the earliest time an instruction becomes non-speculative and is considered safe to execute regardless of its effects on $\mu$-architectural state~\cite{sakalis2019efficient}.
%Instructions that can cast speculative shadows on younger instructions in the dynamic instruction stream are known as \textbf{speculation primitives}.
Speculative shadows can be of the following types: \emph{E-Shadows} are cast by any instruction that can cause an \textbf{exception}; 
\emph{C-Shadows} are cast by \textbf{control instructions}, such as branches and jumps, when either the branch condition or the target address are unknown or have been predicted but not yet verified; \emph{D-Shadows} are cast by potential \textbf{data dependencies} through stores with unresolved addresses (read-after-write dependencies); \emph{M-shadows} are cast by \textbf{speculatively executed memory accesses} that may be caught violating the ordering rules of a memory model (e.g., total store order---TSO) and therefore may need to be squashed; and \emph{V-shadows} that are cast by \textbf{value-predicted loads}~\cite{sakalis2019efficient}.

\subsection{Delay-on-Miss and Memory Consistency Models}
\label{sec:dom-vp}
%\input{dom-vp} We do not need more sub-levels do we?
The goal of Delay-on-miss is to hide speculative changes in the memory hierarchy (including DRAM). 
To achieve this, Delay-on-miss delays speculative loads that miss in the L1. Loads that hit in the L1 (and their dependent instructions) are allowed to execute speculatively as their effects (i.e., on the L1 replacement state) can be deferred for when the loads are cleared from any speculation shadow. 
The miss of a delayed load is allowed to be resolved in the memory hierarchy at the earliest point the load becomes non-speculative. An efficient mechanism to track shadows is proposed in~\cite{sakalis2019efficient}. 

A critical factor that affects the performance of Delay-on-miss (with respect to a non-delaying baseline) is its effect on MLP for the delayed misses. 
A stalled L1 miss is allowed to proceed (and affect changes in the $\mu$-architectural state) just after the \emph{first} load that caused the miss becomes non-speculative.
Many times, a single L1 miss satisfies multiple loads.\footnote{This property was exploited since the early high-performance architectures such as the Alpha 21164: each of its MSHRs could satisfy up to four outstanding loads that missed on the same cacheline~\cite{alpha21164}.} 
In such cases, MLP among loads to the same cacheline is attained effortlessly.
MLP, however, among loads on \emph{different} cachelines is more involved and depends on restrictions imposed by the memory consistency model (MCM). We will examine TSO and {\rc} as representative MCMs.

Assume that a delayed load exits all speculative shadows: E-Shadows (no instruction before it, or itself, will cause an exception), C-Shadow (all preceding branches have resolved), and D-Shadow (all preceding stores have resolved their addresses). The only shadow that is left is the M-Shadow that is governed by the MCM:
\squishlist
\item {\rc}: In {\rc} load$\rightarrow$load order is not enforced. An unperformed load does not cast an M-Shadow on younger loads, therefore there is no restriction on the MLP among delayed loads that are unshadowed.
\item TSO: In TSO load$\rightarrow$load order \emph{is} enforced and unperformed loads cast M-Shadows on younger loads, preventing them from releasing their L1 misses. However, even in this case there is a way to remove the M-Shadow. The key is non-speculative load-load reordering proposed by Ros et al.~\cite{aros-isca17}. Assuming that a load is unshadowed by any other shadow other than the M-Shadow, the M-Shadow can be removed by making the younger loads non-speculative via the cache protocol~\cite{aros-isca17}. \emph{Under this perspective TSO is no different than {\rc} with regards to Delay-on-miss MLP.}
\squishend

To conclude, one can argue that we do not need M-Shadows in any MCM: {\rc} does not have M-Shadows, and a TSO implementation that allows non-speculative load--load reordering~\cite{aros-isca17}, effectively eliminates the M-Shadows. 

Non-speculative load--load reordering delays remote stores which can be construed as a side-channel of a similar threat level to an invalidation side-channel~\cite{}. However, we stress here that this behavior is not a \emph{speculative} side channel as it only takes effect for \emph{unshadowed} loads, after they release their misses to the memory system.

\subsection{Delay-on-miss and Value Prediction} 

%VP needs to be validated. In order! Non-speculative load-load reordering cannot hep here as we are bound by a new speculation: VP-speculation (VP-Shadow). This prevents any MLP for the validation phase. Whatever VP wins in latency (pre-executing load and dependent instructions if correct), looses later in MLP. Result: limited.
In Delay-on-miss, the wast majority of all loads are executed speculatively (80+\% on average~\cite{}), which causes a notable fraction of the loads to be delayed. This takes up precious resources (i.e., the instruction queue, reorder buffer, load./store queue) and prevents instructions to commit. Similar issues arise with other ``delay'' approaches such as NDA~\cite{} and STT~\cite{}.
The amount of speculation results in each load being covered by several speculative shadows (five on average~\cite{}). This forces the majority of the loads to be executed serially, severely limiting MLP. Furthermore, removing any individual shadow (e.g., the M-Shadow) has a limited effect, as the load will be covered by another overlapping shadow.

VP cannot help much in this case, as it only provides values early. The validation is still delayed until all shadows have been lifted. Thus, precious resources are still occupied until the same point in time as only delaying.
The only perceptible difference is a faster commit of pre-executed dependent instructions if the validation of a VP proves to be correct.

VP however introduces a new speculative shadow, which we refer to as \emph{V-Shadow}. This new shadow is only lifted off from younger loads when the validation of the VP is complete. Thus, not only VP occupies precious resources in the same manner as the baseline Delay-on-miss, but further prevents younger loads from validating in parallel. The astute reader will note that this is the same as the M-Shadow an unperformed load casts on younger loads. Therefore a V-Shadown should not mater in the cases where we do have M-Shadows as the two shadows would completely overlap. As we argued above, we can eliminate the M-Shadow (even in MCMs that enforce load$\rightarrow$load ordering) which leaves the \emph{V-Shadow} as a liability that detracts from MLP and performance.

\subsection{Value Recomputation}
\label{sec:recmp}
%%%%%%%%%%%%%%%%%%%%%%%%%%%%%%%%%%%%%%%%%%%%%%%%%%%%%%%%%%%%%
\ignore{
--Both Spectre and Meltdown exploit speculative meta-data (generated by speculative loads).
--Target: Speculative code that makes *any* change in microarchitectural state, specifically cache hierarchy
    -- "Any futuristic attack caused by a speculative load" [InvisiSpec]
    -- What todo about stores?
    -- Does the type of load speculation matter? 
        --Mem consistency violation (depends on consistency model)
            -- A load reading data (i.e., being performed) before other loads or stores --preceding the load in program order-- are performed, may cause such violations...
        --Instr. at ROB head squashed (due to an exception e.g.), load was subsequent...
        --Branch misprediction (preceding branch)
        --St with address speculation; once address is calculated and matches a later load's (which already read data speculatively).
    -- We make speculation invisible in the data-cache, similar to InvisiSpec...
    -- We assume that everything operates correctly
%
    \begin{verbatim}
    Summarize Amnesiac
        [u-arch details go to main sect.]
    Put into perspective 
        "Security-aware" recomputation
        Cmp & Contrast
            Slice generation policy
        Vulnerabilities?
    \end{verbatim}
} We do not need more sub-levels do we?

{\color{red} To be completed by the Amnesiac team}

{\color{blue} Proposed as a way to trade in-core computation (cheap in energy) to data movement (expensive in energy) and at the same time perform better when the latency to re-compute is less than the latency to move data throughout the memory hierarchy.
\paragraph{Why RC can offer beyond VP}
Benefits in both latency and MLP: Latency: if recomputation is faster than a miss. MLP: No need to validate, hence does not impose any further restrictions on MLP, hence better performance. Energy: additional benefits in energy if recomputation needs less energy than data movement. Limitation: cannot do too much of it in practice. }

{\color{red} Magnus notes:
This can be observed for the case where an \emph{oracle} VP is used. One might think that a VP that has 100\% coverage and accuracy would perform equally to the recomputation oracle since the load value in both cases is provided immediately. But this is not the case. The vp-100 does not provide any significant improvement compared to the ISCA VP with ~20\% coverage (see nomralizedpc.pdf). For an improvement instant validation (no delay) is needed. VP is basically limited by the same "property" as delay-on-miss.

Recomputation alleviates this problem by removing the need to delay. A recomputed value is "instantly" (RC slice latency) available and can be committed as soon as all the shadows are lifted (in contrast to delay/VP, which requires a load/validation to be performed before commit).
}



 