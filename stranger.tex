\begin{verbatim}
* Stranger Things go here for the moment 
until we figure out what to do with them.
 \end{verbatim}


\subsection{Coherence and Consistency}
On TSO:
RC seems to work wonders. This is because when we recompute,
the result is by definition correct --- we do not verify.
In this respect TSO is not violated by something that cannot change (see
the ISCA'17 "Non-speculative Load-Load reordering in TSO").
The situation is clear if the recomputation starts off from constants to
compute a new value. The new value is immutable by other cores.
That's what TSO wants.

If the recomputation starts off from values loaded from the L1 (e.g.,
hits) then we must make sure that these values do not change.
But that's easy to do: we must ensure lock-downs (ISCA'17) for these
loads. We can discuss how secure is that, but I believe we can claim
that we can do RC that respects TSO even in this case. 

When we use a real RC, the compiler detects the posibility of recomputation if all the inputs are constants (or pseudo-constant (write-once)?). In this case TSO is correct. The compiler is giving us the guarantee.

The problem may come from 2 sources:

- 1. The compiler does not know well (may-alias) if an input is constant.

- 2. Parallel applications make it worse the first case.

The idea could be to "predict at compiler time" no alias, and add ISCA'17 checks and guarantees at runtime.

Right now, I would go for our current solution. In case of alias we do not recompute(we lose coverage), and we do not add speculation there. -- We can speculate in sw and fix it in hw for a follow up paper.

Note: the compiler cannot guarantee DRF for x86 code, as the model is release consistency instead. 


